\documentclass[11pt]{article}
\usepackage[margin=1in]{geometry}
\usepackage{fancyhdr}
\usepackage{listings}
\setlength{\parindent}{0pt}
\setlength{\parskip}{5pt plus 1pt}
\setlength{\headheight}{13.6pt}
\newcommand\question[2]{\vspace{.25in}\hrule\textbf{#1: #2}\vspace{.5em}\hrule\vspace{.10in}}

\renewcommand\part[1]{\vspace{.10in}\textbf{(#1)}}

\pagestyle{fancyplain}
\lhead{\textbf{\NAME}}
\rhead{Distributed Systems - Project 1, \today}
\begin{document}\raggedright

\newcommand\NAME{Lukas, Young, Vincent}  % your name

\question{1}{Sensor Framework}

\part{A-a}

\begin{small}
\begin{lstlisting}[frame=single]

SensorManager sensorManager = (SensorManager) getSystemService(SENSOR_SERVICE);
List<Sensor> sensors = sensorManager.getSensorList(Sensor.TYPE_ALL);
\end{lstlisting}
\end{small}

\part{A-b}

Assuming sensor references the Sensor we are interested in:

\begin{small}
\begin{lstlisting}[frame=single]
float maxRange = sensor.getMaximumRange();
\end{lstlisting}
\end{small}


\part{A-c}

\begin{small}
\begin{lstlisting}[frame=single]
Sensor accSensor = sensorManager.getDefaultSensor(Sensor.TYPE_ACCELEROMETER);
registerListener (listener, accSensor, SENSOR_DELAY_FASTEST, handler)
\end{lstlisting}
\end{small}

Where sensorManager, accSensor and handler have been defined appropriately.

\part{B}

The SensorEvent objects passed to the SensorEventListener can be resused by the system. This means that the values array may be overwritten by the system to hold data for some other event. It is thus necessary to copy the values in the onSensorChanged method, and pass that copy to the listenerActivity.

\question{2}{Activity Lifecycle} 
\part{a} An activity A is in the foreground, then the user stars another activity B.
 
 -Activity B is created and onCreate() is called
 
 -Activity B is started and onStart() is called
 
 -Activity A is paused and onPause() is called
 
 -Activity B is resumed and onResume() is called

\part{b} Activity A is no longer visible.

- Activity A is stopped and onStop() is called.

\part{c} User navigates back to Activity A.

There are two cases:

Activity A had been destroyed by the system:

- That means Activity A was destroyed before and OnDestroy was called.

(Depending if the system destroys it because it is not needed anymore or because the memory is needed, the System will save the state of the activity.)

-Activity A is created and onCreate() is called

-Activity A is started and onStart() is called

-Activity A is resumed and onResume() is called

The activity A was only stopped:

-Activity A is restarted and onRestart() is called

-Activity A is started and onStart() is called

-Activity A is resumed and onResume() is called.

In any case B will get paused, and  - once/if it is covered - stopped.

\question{3}{Resources}

The Android XML Layouts can be defined so that interface elements are layed out relative to each other and the screen border. This is good for small variations in screen size. For vastly different screen sizes and densities different layouts can be defined. The corresponding activity can then instanciate the most appropriate one in its onCreate.

Another way to do it is to instaciate all Views programmatically and make this layout and drawing code work in consideration of the available screen realestate. But often doing things all programmatically is more tedious.

Android XML elements can have an id. An element with id $x$ can be referenced in code by $R.id.x$.

\question{4}{Intents} 

\textbf {What are Intents?}

An Intent is a messaging object and an abstract description of an operation to be performed.

\textbf{What are they used for? }

You can use an Intent to request an action from another app component and carrying any necessary data to it.
There are basic cases:\newline
1. To start an activity
	
It will start the window which is an instance of the Activity. 
The Activity can then use the data that got carried over and perform its own methods.
One can use startActivityForResult() to receive a result when the Activity finishes.
 
2. To start a service

This will start a service which will perform its operations with help of the data carried over.
Compared to the activity it will perform in the background without an user interface.

3. To deliver a broadcast

This will basically send a message which any app can receive.

\textbf {What is the difference between Explicit Intents and Implicit Intents?}

\textbf{Explicit Intents:}\newline
- For an Explicit Intent you have to specify the component to start by a fully-qualified class name.

- Typically you use an Explicit Intent to start a component in your own app.

- For 1. and 2. from the question before the system will immediately start the app component communicated in the Intent.

\textbf{Implicit Intents:}\newline
- Compared to the Explicit Intents - An Implicit Intent will will not name a specific app component like an action or service.

- Instead it will declare a general action which is allowed to be handled by an component of    a completely different app. 	

- To find out which components are the appropriate ones the system will check the Intent filters which should be declared in the manifest file.
If it has a match then the system starts the match. If there are more than one matches the user can pick from the list.
And if none are found the call will fail and the app will crash if one did not verify before that there is at least one match.\newline
\textbf{Note:} The intent-Filter does not influence any explicit Intents



\question{5}{Service Lifecycle} 

\part{a} 

\textbf{false:} A service can be stopped with the method stopService().

\part{b}

\textbf{false:} Only bound services offer an interface to communicate with multiple processes.

\part{c}

\textbf{true:} From the API Guide: "A bound service runs only as long as another application component is bound to it. Multiple components can bind to the service at once, but when all of them unbind, the service is destroyed."

\part{d}

\textbf{false:} To quote the API Guide "Caution: A service runs in the main thread of its hosting process - the service does not create its own thread and does not run in a separate process (unless you specify otherwise).".

\question{6}{AndroidManifest file}

 \textless uses-permission android:name= "android.permission.WRITE\_SMS"/\textgreater 
 
 
 \textless uses-permission android:name="android.permission.ACCESS\_FINE\_LOCATION"/\textgreater 
 
 
 \textless uses-permission android:name="android.permission.READ\_PHONE\_STATE" /\textgreater 
 
 \textless service\textgreater 
 
 	android:name="ch.ethz.inf.vs.android.nethz.antitheft.LocationService"
	 
 \textless /service\textgreater 

\end{document}