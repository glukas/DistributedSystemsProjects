\documentclass[11pt]{article}
\usepackage[margin=1in]{geometry}
\usepackage{fancyhdr}
\setlength{\parindent}{0pt}
\setlength{\parskip}{5pt plus 1pt}
\setlength{\headheight}{13.6pt}
\newcommand\question[2]{\vspace{.25in}\hrule\textbf{#1: #2}\vspace{.5em}\hrule\vspace{.10in}}

\renewcommand\part[1]{\vspace{.10in}\textbf{(#1)}}

\pagestyle{fancyplain}
\lhead{\textbf{\NAME}}
\rhead{Distributed Systems - Project 1, \today}
\begin{document}\raggedright

\newcommand\NAME{Lukas, Young, Vincent}  % your name

\question{5}{Service Lifecycle} 

\part{a} 

\textbf{wrong}, a service can be stopped with the method stopService().

\part{b}

\textbf{wrong}, we call "bound service" a service that have at least one activity is bonded to it. Thus, an unbound service doesn't interact with client process.

From android.developper.com :

"A service is "bound" when an application component binds to it by calling bindService()."

\part{c}

\textbf{wrong}, if someone has called startService(), service is only destroy after a call of stopService() or stopSelf() and all clients have called onUnbind().

\part{d}

\textbf{true}, services are always called in a separate thread.

\question{6}{AndroidManifest file}

\part{1}

 \textless uses-permission android:name= "android.permission.WRITE\_SMS"/\textgreater 

\part{2}

 \textless uses-permission android:name="android.permission.ACCESS\_FINE\_LOCATION"/\textgreater 

\part{3}

 \textless uses-permission android:name="android.permission.READ\_PHONE\_STATE" /\textgreater 

\end{document}