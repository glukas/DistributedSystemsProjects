\documentclass[11pt]{article}
\usepackage[margin=1in]{geometry}
\usepackage{fancyhdr}
\usepackage{listings}
\setlength{\parindent}{0pt}
\setlength{\parskip}{5pt plus 1pt}
\setlength{\headheight}{13.6pt}
\newcommand\question[2]{\vspace{.25in}\hrule\textbf{#1: #2}\vspace{.5em}\hrule\vspace{.10in}}

\renewcommand\part[1]{\vspace{.10in}\textbf{(#1)}}

\pagestyle{fancyplain}
\lhead{\textbf{\NAME}}
\rhead{Distributed Systems - Project 1, \today}
\begin{document}\raggedright

\newcommand\NAME{Lukas, Young, Vincent}  % your name

\question{1}{Sensor Framework}

\part{A-a}

\begin{small}
\begin{lstlisting}[frame=single]

SensorManager sensorManager = (SensorManager) getSystemService(SENSOR_SERVICE);
List<Sensor> sensors = sensorManager.getSensorList(Sensor.TYPE_ALL);
\end{lstlisting}
\end{small}

\part{A-b}

\begin{small}
\begin{lstlisting}[frame=single]

Sensor sensor = ...;
float maxRange = sensor.getMaximumRange();
\end{lstlisting}
\end{small}

\part{A-c}

\begin{small}
\begin{lstlisting}[frame=single]
Sensor accSensor = sensorManager.getDefaultSensor(Sensor.TYPE_ACCELEROMETER);
registerListener (listener, accSensor, SENSOR_DELAY_FASTEST, handler)
\end{lstlisting}
\end{small}

Where sensorManager, accSensor and handler have been defined appropriately.

\part{B}

(Note: I am still unsure about this. The reusing is only a problem if the MainActivity also doesn't copy the accelerometer data. What do you think? )

The SensorEvent objects passed to the SensorEventListener can be resused by the system. This means that the values array may be overwritten by the system to hold data for some other event. It is thus necessary to copy the values in the onSensorChanged method, and pass that copy to the listenerActivity.

As a sidenote, it is unadvisable to block in the onSensorChanged method. The main activity should thus not directly update its display, but rather store the new values invalidate the views.

\question{3}{Resources}

The Android XML Layouts can be defined so that interface elements are layed out relative to each other and the screen border. This is good for small variations in screen size. For vastly different screen sizes and densities different layouts can be defined. The corresponding activity can then instanciate the most appropriate one in its onCreate.

Another way to do it is to instaciate all Views programmatically and make this layout and drawing code work in consideration of the available screen realestate. But often doing things all programmatically is more tedious.

Android XML elements can have an id. An element with id $x$ can be referenced in code by $R.id.x$.


\question{5}{Service Lifecycle} 

\part{a} 

\textbf{wrong}, a service can be stopped with the method stopService().

\part{b}

\textbf{wrong}, we call "bound service" a service that have at least one activity is bonded to it. Thus, an unbound service doesn't interact with client process.

From android.developper.com :

"A service is "bound" when an application component binds to it by calling bindService()."

\part{c}

\textbf{wrong}, if someone has called startService(), service is only destroy after a call of stopService() or stopSelf() and all clients have called onUnbind().

\part{d}

\textbf{true}, services are always called in a separate thread.

\question{6}{AndroidManifest file}

\part{1}

 \textless uses-permission android:name= "android.permission.WRITE\_SMS"/\textgreater 

\part{2}

 \textless uses-permission android:name="android.permission.ACCESS\_FINE\_LOCATION"/\textgreater 

\part{3}

 \textless uses-permission android:name="android.permission.READ\_PHONE\_STATE" /\textgreater 

\end{document}