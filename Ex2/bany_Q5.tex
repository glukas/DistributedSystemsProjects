\documentclass[11pt]{article}
\usepackage[margin=1in]{geometry}
\usepackage{fancyhdr}
\usepackage{listings}
\setlength{\parindent}{0pt}
\setlength{\parskip}{5pt plus 1pt}
\setlength{\headheight}{13.6pt}
\newcommand\question[2]{\vspace{.25in}\hrule\textbf{#1: #2}\vspace{.5em}\hrule\vspace{.10in}}

\renewcommand\part[1]{\vspace{.10in}\textbf{(#1)}}

\pagestyle{fancyplain}
\lhead{\textbf{\NAME}}
\rhead{Distributed Systems - Project 2, \today}
\begin{document}\raggedright

\newcommand\NAME{Lukas, Young, Vincent}  % your name

\question{4}{WS-* services} 
\part{a}  
\textbf{Which document holds information about the definition of the SunSPOTWebservice?}\newline
The WSDL document located at: \newline
http://vslab.inf.ethz.ch:8080/SunSPOTWebServices/SunSPOTWebservice?WSDL \newline
holds the information about the SunSPOTWebservice. \newline

\textbf{How can this document be retrieved?}\newline
You can retrieve the document by accessing: \newline
http://vslab.inf.ethz.ch:8080/SunSPOTWebServices/SunSPOTWebservice?WSDL .


\part{b} 
\textbf{Where can the type definition of the elements getSpot and getSpotResponse be found?}\newline
You can find the type definitions of the elements getSpot and getSpotResponse by accessing:\newline http://vslab.inf.ethz.ch:8080/SunSPOTWebServices/SunSPOTWebservice?xsd=1 .

\textbf{Give the element definitions for both getSpot and getSpotResponse.}\newline

getSpot: 
\begin{small}
\begin{lstlisting}[frame=single]
<xs:complexType name="getSpot">
<xs:sequence>
<xs:element name="id" type="xs:string" minOccurs="0"/>
</xs:sequence>
</xs:complexType>
\end{lstlisting}
\end{small}

getSpotResponse:
\begin{small}
\begin{lstlisting}[frame=single]
<xs:complexType name="getSpotResponse">
<xs:sequence>
<xs:element name="return" type="tns:sunSpot" minOccurs="0"/>
</xs:sequence>
</xs:complexType>
\end{lstlisting}
\end{small}


\part{c} \textbf{Imagine, the SunSPOTWebService would be implemented using SMTP as transport protocol. Where in the WSDL file would you declare the transport protocol? }\newline
We would declare it in the line which contains : \newline
\textless soap:binding transport="http://schemas.xmlsoap.org/soap/http" style="document"/\textgreater \newline
right now. \newline
Instead we would write: \newline
\textless soap:binding transport="http://webservices.vslecture.vs.inf.ethz.ch/smtp" style="document"/\textgreater \newline

\textbf{How does this affect the soap:address in the service definition?}\newline
We would have to change: \newline
\textless soap:address location="http://vslab.inf.ethz.ch:8080/SunSPOTWebServices/SunSPOTWebservice"/\textgreater\newline
 to \textless soap:address location="mailto:suscribe@webservices.vslecture.vs.inf.ethz.ch"/\textgreater




\question{5}{Android Emulator Networking} 

\part{a} \textbf{What IP address is assigned to an emulated device?}\newline
The IP address assigned to an emulated device is: 10.0.2.15\newline

\textbf{Why is it the same address even if multiple emulated instances run on the same development machine?}\newline
The reason for this is because the emulated instance will have its own router and behind each own router the instances will have their own IP address of 10.0.2.15. That means that all the emulated instances are isolated from each other by the router.

\part{b} \textbf{To whom does a call on an emulated instance to 127.0.0.1 refer?}\newline
It refers to the emulator's own loopback interface.


\part{c} \textbf{By which IP address can the development machine be reached from an emulated device?}\newline
The developement machine can be reached by the IP address: 10.0.2.2

\part{d} \textbf{How can the development machine connect to a port on the emulated device?}\newline
You can connect to the console port of a specific emulator instance first with the command: \newline
telnet localhost \textless consoleport\textgreater . (\textless consoleport\textgreater could for example be 5554) 


\end{document}