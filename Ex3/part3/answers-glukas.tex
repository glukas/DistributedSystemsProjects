\documentclass[11pt]{article}
\usepackage[margin=1in]{geometry}
\usepackage{fancyhdr}
\usepackage{listings}
\usepackage{graphicx}
\setlength{\parindent}{0pt}
\setlength{\parskip}{5pt plus 1pt}
\setlength{\headheight}{13.6pt}
\newcommand\question[2]{\vspace{.25in}\hrule\textbf{#1: #2}\vspace{.5em}\hrule\vspace{.10in}}

\renewcommand\part[1]{\vspace{.10in}\textbf{(#1)}}

\pagestyle{fancyplain}
\lhead{\textbf{\NAME}}
\rhead{DS - Project 3, \today}
\begin{document}\raggedright

\newcommand\NAME{Lukas Gianinazzi, Young Ban, Vincent Demotz}  % your name

\question{3}{Mini-Test}

\part{1}
What are the main advantages of using Vector Clocks over Lamport timestamps?


\begin{itemize}
	\item Vector Clocks keep tracks of causal dependencies
	\item Vector Clocks fulfill strong clock condition
\end{itemize}

\part{2}
Give the conditions for two Vector Clocks to be causally dependent?

\begin{itemize}
\item$e^{x}_{i} \Rightarrow e^{x}_{j}$ iff $C_{i} (e^{x}_{i})[i] \leq C_{j} (e^{y}_{j})[i] \wedge 
\neg ( C_{j} (e^{y}_{j})[i] \leq C_{i} (e^{x}_{i})[i])$
\end{itemize}

\clearpage

\part{3}
Draw the state machine for a successful registration. For this, identify the commands on top of
arrows and link the states to your own naming in your implementation. (Hint: What if there is no
network connection?)


\includegraphics[scale=0.15]{registrationStates.jpg}

\textbf{Figure 1, states machine for registration}

\begin{itemize}
	\item \textbf{State 1 : } App waits for an instruction from user
	\item \textbf{State 2 : } App waits for reply from network
	\item \textbf{State 3 : } Registration succeeded, can launch MainActivity
	\item \textbf{State 4 : } App quits
	\item \textbf{$L_{1}$ : } App is launched
	\item \textbf{$L_{2}$ : } App process is destroyed
	\item \textbf{$L_{3}$ : } MainActivity is created and started
	\item \textbf{$A_{1}$ : } onBackPress() is called
	\item \textbf{$A_{2}$ : } onLogin() is called and haveNetworkConnection() returns true
	\item \textbf{$A_{3}$ : } onLogin() is called and haveNetworkConnection() returns false, display AlertDialog to inform user
	\item \textbf{$B_{1}$ : } onRegistrationFailed() is called
	\item \textbf{$B_{2}$ : } onRegistrationSucceeded() is called
\end{itemize}

\clearpage

\part{4}
Draw the state machine for sending a message.

\includegraphics[scale=0.15]{sendMessageStates.jpg}

\textbf{Figure 2, states machine for send message}

\begin{itemize}
	\item \textbf{State 1 : } App waits for user to enter message
	\item \textbf{State 2 : } App waits for reply from network about the message just send
	\item \textbf{State 3 : } Message has been successfully delivered
	\item \textbf{State 4 : } Message failed to be delivered
	\item \textbf{$L_{3}$ : } User has been successfully registered
	\item A : sendMessage(String text, int id) is called on ChatLogic
	\item \textbf{$B_{1}$ : } onMessageDeliverySucceeded is called from ChatLogic
	\item \textbf{$B_{2}$ : } onMessageDeliveryFailed is called from ChatLogic
	\item \textbf{$R_{1}$ : } calls setTryToSendToFailure(int IdProvided)
	\item \textbf{$R_{2}$ : } calls setTryToSendToSend(int IdProvided)
\end{itemize}

\clearpage

\part{5}
We decided in the exercise that we would not let our applications trigger a tick when receiving a
message. What would be the implications of ticking on receive?


\begin{itemize}
	\item It would be possible to sort receiving events.
\end{itemize}

\part{6}
Does a clock tick happen before or after the sending of a message. What are the implications of
changing this?


\begin{itemize}
	\item Before. Else it would be possible to violates clock consistency condition. For example, a client A receives a message m1 with timestamps 0. Client sends a message m2 with time 0 and then tick. Clearly, we have m1 $\rightarrow$ m2 and it should imply that C(m1) $<$ C(m2), but both have timestamps 0.

\end{itemize}

\part{7}
Read the paper Dynamic Vector Clocks for Consistent Ordering of Events in Dynamic Distributed
Applications by Tobias Landes 8
that gives a good overview on the discussed methods. In particular,
which problem of vector clocks is solved in the paper?


\begin{itemize}
	\item The problem with vector clocks is that one needs to know the number of processes in advance and this number has to be fix. The paper solves the problem in proposing a solution where one has possibilities to create dynamically processes at runtime.
\end{itemize}














\end{document}